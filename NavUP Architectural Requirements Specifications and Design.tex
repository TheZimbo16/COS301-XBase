\documentclass{article}
\usepackage[utf8]{inputenc}

\title{Architectural Requirements Specifications and Design}
\author{Andrew le Roux \hfill [Student Number] \\ Cian Steenkamp \hfill [15095682] \\ Darren Adams \hfill [Student Number] \\ Drew Langley \hfill [Student Number] \\ Martha Mohlala \hfill [Student Number] \\ Nsovo Baloyi \hfill [Student Number] \\ Rikard Schouwstra \hfill [Student Number]}
\date{March 2017}
\begin{document}
\maketitle
\pagebreak
\tableofcontents
\pagebreak
\section{External Interface Requirements}
    \subsection{User Interfaces}
        The mobile application will interface with the supported input and output
        features of the host's operating system. Inputs include text that the user
        will enter for login or searching a venue. Outputs include the type of fonts
        to display text or graphics to show images or draw the map.

    \subsection{Hardware Interfaces}
        Since neither the mobile application nor the web portal have any designated
        hardware, it does not have any direct hardware interfaces. The WiFi software
        in the mobile phone manages the built-in WiFi and the hardware connection
        to the database server is managed by the underlying operating system on the
        mobile phone and the web server.

    \subsection{Software Interfaces}
        The mobile application communicates with the WiFi software in order to get
        signal strength information from multiple WiFi access points to determine
        (using triangulation) where the user is located. The communication software
        between the database and mobile application consists of operation concerning
        creating, reading, removing and modifying the data.

    \subsection{Communication Interfaces}
        The communication between the different parts of the system are important since they depend on each other. However, in what way the communication is achieved is not important for the system and is therefore handled by the underlying operating systems for both the mobile application and the back-end of the system.

\section{Performance Requirements}
\section{Design Constraints}
\section{System Software Attributes}

    \subsection{Reliability}
        
        \begin{description}
        
        \item[$\bullet$] Any information that is stored on the database must remain correct when being transferred to the user interface.
        \item[$\bullet$] The services offered by the system should be available to users except for when the system is undergoing maintenance.
        \item[$\bullet$] The system should reply to user requests in the shortest time interval possible.
        \item[$\bullet$] The system must be fault tolerant, it needs to maintain a certain level of performance and offer other services that are not affected by this fault to the users.
        \item[$\bullet$] In the event of a fault the system must be able to recover within the shortest time period possible and recover any data that may have been lost.
        \item[$\bullet$] The system should be able to respond appropriately if it receives bad input data from the user.
        
        \end{description}
        
    \subsection{Scalability}
    
        \begin{description}
    
        \item[$\bullet$] The system must be able to cater for increases in the work load, for example large number of users or activities at any given time, without impacting the performance of the system. 
        \item[$\bullet$] If the system does not cater for increases in workload it should at least provide the ability to be readily enlarged.
        
        \end{description}
    
    \subsection{Maintainability}
    
        \begin{description}
    
        \item[$\bullet$] The system must be designed in a modular fashion that provides high cohesion and loose coupling, this will allow parts of the system to be easily maintained without affecting the rest of the system.
        \item[$\bullet$]Maintenance should be able to be carried out by different maintenance teams, therefore the system must be easy to learn and understand.
        
        \end{description}
    
    \subsection{Integrability}
    
        \begin{description}
        
        \item[$\bullet$] Since we are following a modular design, components of the system that are separately developed should work correctly together.
        \item[$\bullet$] Follow coding standards specified by the client to allow for easy integration and employ continuous integration in our design process.
        
        \end{description}
    
    \subsection{Usability}
    
        \begin{description}
        
        \item[$\bullet$] The system must be easy to learn.
        \item[$\bullet$]System must cater for user mistakes, by providing the user with the undo or roll back options.
        \item[$\bullet$]The user interface must be easy to use and must be intuitive.
        \item[$\bullet$]The system should display options in a logical manner.
        \item[$\bullet$]Incorporate widgets and icons that the target users may be familiar with.
        \item[$\bullet$]The user manual should have a detailed description of the system.
        \item[$\bullet$]A help option must be provided to the users.
        
        \end{description}
    
    \subsection{Interoperability}
    
        \begin{description}
        
        \item[$\bullet$]The system must be able to communicate with the University of Pretoria WiFi system, because the WiFi  access points will be used for the navigation.
        
        \end{description}

\section{Modules}
    \subsection{User Management}
        \subsubsection{Deployment Diagram}
        \subsubsection{Class Diagram}
        \subsubsection{Use Case Diagrams}
        \subsubsection{Other Diagrams}
        \subsubsection{Chosen Design Patterns}
    \subsection{Notifications}
        \subsubsection{Deployment Diagram}
        \subsubsection{Class Diagram}
        \subsubsection{Use Case Diagrams}
        \subsubsection{Other Diagrams}
        \subsubsection{Chosen Design Patterns}
    \subsection{Navigation}
        \subsubsection{Deployment Diagram}
        \subsubsection{Class Diagram}
        \subsubsection{Use Case Diagrams}
        \subsubsection{Other Diagrams}
        \subsubsection{Chosen Design Patterns}
    \subsection{Points-of-interest}
        \subsubsection{Deployment Diagram}
        \subsubsection{Class Diagram}
        \subsubsection{Use Case Diagrams}
        \subsubsection{Other Diagrams}
        \subsubsection{Chosen Design Patterns}
\section{Technology Choices}

\end{document}